\documentclass[11pt]{article}
\usepackage[utf8]{inputenc}
\usepackage{hyperref}
\usepackage{xcolor}
\usepackage{listings}
\usepackage{enumitem}
\usepackage[margin=1in]{geometry}
\definecolor{lightgray}{gray}{0.95}

\lstset{
	backgroundcolor=\color{lightgray},
	basicstyle=\ttfamily\small,
	frame=single,
	breaklines=true
}

\title{2a Adding a Fly-squito - Banana Mayhem}
\author{Nicholas \& Aikens}
\date{\today}

\begin{document}
	\maketitle
	
	\section{FlyMovement.cs}
	
	\textbf{Purpose:} Chases the player and triggers a “death” (scene reload) on contact.
	
	\begin{enumerate}[label=\arabic*.]
		\item In \texttt{Start()}, find and cache the \texttt{Transform} of the GameObject tagged “Player”.  
		\item In \texttt{Update()}, compute the normalized direction vector toward the player and translate the fly each frame by \texttt{speed * Time.deltaTime}.  
		\item In \texttt{OnTriggerEnter2D()}, detect collision with the player and reload the active scene.  
	\end{enumerate}
	
	\begin{lstlisting}[language=C]
		// FlyMovement.cs
		using UnityEngine;
		using UnityEngine.SceneManagement;
		
		[RequireComponent(typeof(Rigidbody2D))]
		public class FlyMovement : MonoBehaviour
		{
			[Tooltip("Chase speed in units per second")]
			public float speed = 3f;
			
			private Transform player;
			
			void Start()
			{
				// Find the Player by tag in the scene
				GameObject playerGO = GameObject.FindWithTag("Player");
				if (playerGO != null)
				player = playerGO.transform;
				else
				Debug.LogError("No GameObject with tag 'Player' found.");
			}
			
			void Update()
			{
				if (player == null) return;
				
				// Compute direction toward player
				Vector2 direction = (player.position - transform.position).normalized;
				
				// Move the fly
				transform.Translate(direction * speed * Time.deltaTime);
			}
			
			void OnTriggerEnter2D(Collider2D other)
			{
				if (other.CompareTag("Player"))
				{
					// Reload current scene => player dies
					SceneManager.LoadScene(
					SceneManager.GetActiveScene().buildIndex);
				}
			}
		}
	\end{lstlisting}
	
	\section{FlySpawner.cs}
	
	\textbf{Purpose:} Instantiates one fly at each designated spawn point when the scene starts.
	
	\begin{enumerate}[label=\arabic*.]
		\item Expose a \texttt{GameObject flyPrefab} field for the Fly prefab.  
		\item Expose a \texttt{Transform[]} array of spawn points.  
		\item In \texttt{Start()}, loop through each spawn point and \texttt{Instantiate} the prefab at its position.  
		\item Log a warning if no prefab or spawn points are assigned.  
	\end{enumerate}
	
	\begin{lstlisting}[language=C]
		// FlySpawner.cs
		using UnityEngine;
		
		public class FlySpawner : MonoBehaviour
		{
			[Tooltip("Assign your Fly prefab here")]
			public GameObject flyPrefab;
			
			[Tooltip("These Transforms mark where flies will appear")]
			public Transform[] spawnPoints;
			
			void Start()
			{
				if (flyPrefab == null || spawnPoints.Length == 0)
				{
					Debug.LogWarning(
					"FlySpawner needs a prefab and at least one spawn point.");
					return;
				}
				
				foreach (Transform point in spawnPoints)
				{
					Instantiate(flyPrefab, point.position, Quaternion.identity);
				}
			}
		}
	\end{lstlisting}
	
	\section{Scene Setup}
	
	\begin{enumerate}[label=\arabic*.]
		\item \textbf{Player GameObject}:
		\begin{itemize}
			\item Must be tagged \texttt{"Player"}.
			\item Should have a \texttt{Collider2D} (e.g., \texttt{CircleCollider2D}) and a \texttt{Rigidbody2D}.
		\end{itemize}
		\item \textbf{Fly Prefab}:
		\begin{itemize}
			\item Add your fly sprite as a GameObject.
			\item Attach a \texttt{CircleCollider2D} (set \texttt{Is Trigger} = true).
			\item Attach a \texttt{Rigidbody2D} (Body Type = \texttt{Kinematic}, Gravity Scale = 0).
			\item Attach the \texttt{FlyMovement.cs} script.
			\item Tag the prefab (or its instances) as \texttt{"Enemy"} if desired.
		\end{itemize}
		\item \textbf{Spawn Points}:
		\begin{itemize}
			\item Create empty GameObjects at positions where you want flies to appear.
			\item Assign these to the \texttt{spawnPoints} array on your \texttt{FlySpawner} component.
		\end{itemize}
		\item \textbf{Fly Spawner}:
		\begin{itemize}
			\item Create an empty GameObject named \texttt{"FlySpawner"}.
			\item Attach the \texttt{FlySpawner.cs} script.
			\item Drag in the Fly prefab and populate the spawn points.
		\end{itemize}
		\item \textbf{Testing}:
		\begin{itemize}
			\item Press Play. Flies should home in on the player.
			\item On contact, the scene reloads, simulating player death.
		\end{itemize}
	\end{enumerate}
	
\end{document}
